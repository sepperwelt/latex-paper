\documentclass[a4paper,fleqn,12pt]{article}
\setlength{\headheight}{23pt}
\usepackage[a4paper, left=2.5cm, right=2.5cm]{geometry}
\usepackage[utf8]{inputenc}
%\usepackage[T1]{fontenc}
\usepackage[ngerman]{babel}
\usepackage{grffile}
\usepackage{eurosym}
\usepackage{graphicx} % Allows including images 
\usepackage{amsmath, amsthm, amssymb}
\usepackage[separate-uncertainty]{siunitx}
\usepackage{ifthen}
\usepackage{fancyhdr}
\pagestyle{fancy}
\usepackage[justification=centering,labelfont=bf]{caption}
\usepackage[bottom]{footmisc}
\usepackage{todonotes}
\presetkeys{todonotes}{inline}{}
\usepackage{tikz}
\usepackage{gnuplot-lua-tikz}
\usepackage{circuitikz}
\usetikzlibrary{calc}
\usepackage{lastpage}
\usepackage{xspace}	
\usepackage[skip=10pt plus1pt, indent=10pt]{parskip}
\usepackage{makecell}
\usepackage{csvsimple}
\usepackage{siunitx}
\usepackage{svg}
\usepackage{sectsty}
\allsectionsfont{\sffamily}
\usepackage{csquotes}
\usepackage{multicol}
\usepackage{setspace}

\usepackage{enumitem}
\setitemize{itemsep=-5pt}
\setenumerate{itemsep=-5pt}
\setlength{\mathindent}{10pt}
\renewcommand{\baselinestretch}{1.5} 

\newcommand\tab[1][0.5cm]{\hspace*{#1}}

\usepackage[					   % Literaturverzeichnis mit "biber [Projektname ohne Endung]" aktualisieren
    backend=biber,
    style=ieee,
]{biblatex}                        % Bibliography management
\addbibresource{references.bib}    % Bibliography file

%%%%%%%%%%%%%%%%%%%%%%%%%%%%%%%%%%%%%%%%%%%%%%%%%%%%%%%%%%%%%%%%%%%%%%%%%%%%%%%%
%%%%%%% PACKAGAES %%%%%%%%%%%%%%%%%%%%%%%%%%%%%%%%%%%%%%%%%%%%%%%%%%%%%%%%%%%%%%
%%%%%%%%%%%%%%%%%%%%%%%%%%%%%%%%%%%%%%%%%%%%%%%%%%%%%%%%%%%%%%%%%%%%%%%%%%%%%%%%

%  Mathematik
\usepackage{amsmath}								%  Mathematikpaket
\allowdisplaybreaks									%  Formeln können auf mehrere Seiten verteilt werden
\usepackage{amssymb}								%  spezielle Mathmematiksymbole
\usepackage{siunitx}								%  korrektes Darstellen von Einheiten
\sisetup{locale = DE}

\usepackage{geometry}								%  Definition der Seitenraender
\usepackage[final]{pdfpages}							% zum einbinden von externen pdf-Seiten
\usepackage{graphicx}								%  zum Einbinden von Grafiken, spziell für dvi->ps Konvertierung
\setlength{\unitlength}{1mm}
\usepackage[hidelinks]{hyperref}	
\usepackage{placeins}								% FloatBarrier
\usepackage{upgreek}
\usepackage{tabularx}								% schöne tabellen
\usepackage{longtable}  								% lange tabellen
%\usepackage[osf,sc]{mathpazo}						% Schriftart	
\usepackage{booktabs}								% linien in der tabelle
\usepackage{nicefrac}								% brüche in zeilen

\usepackage{array}
\newcolumntype{Y}{>{\centering\arraybackslash}X}


% ---------------Mathematisches----------------------------->
%  fett dargestellte Zeichen im Mathematikmodus, Matrizen, Vektoren
\newcommand{\mv}[1]{\ensuremath{\boldsymbol{#1}}}
%  fett dargestellte unterstrichene Zeichen im Mathematikmodus
\newcommand{\kmv}[1]{\ensuremath{\underline{\boldsymbol{#1}}}}
%  Transponierte einer Matrix
\newcommand{\trans}[1]{\ensuremath{\boldsymbol{#1}^{\mathrm{T}}}}
%  Transponierte der konjugiert komplexen Matrix
\newcommand{\herm}[1]{\ensuremath{\boldsymbol{#1}^{\mathrm{H}}}}
%  Rang Matrix
\newcommand{\rg}[1]{\ensuremath{\mathrm{rg}\left({#1}\right)}}
%  Spur einer Matrix
\newcommand{\tr}[1]{\ensuremath{\mathrm{tr}\left\{ #1 \right\}}}
%  Realteil
\newcommand{\re}[1]{\mathrm{Re}\left\{  #1 \right\} }
%  Imaginärteil
\newcommand{\im}[1]{\mathrm{Im}\left\{  #1 \right\} }
%  Unterstreichung ... komplex
\newcommand{\K}[1]{\underline{#1}}
%  erzeugt das "entspricht Zeichen"
\newcommand{\corsp}{\ensuremath{\stackrel{\scriptscriptstyle \bigtriangleup}{=}}}
%  gewöhnliche Ableitung
\newcommand{\ddx}[1][x]{\frac{\mathrm{d}}{\mathrm{d} #1}}
%  partielle Ableitung
\newcommand{\pddx}[1][x]{\frac{\partial}{\partial #1}}
%  zweifache partielle Ableitung
\newcommand{\ppddx}[1][x]{\frac{\partial^2}{\partial #1 ^2}}
%  d-Operator
\newcommand{\dd}{\mathrm{d}}
%  Einheitsmatrix							
\newcommand{\ident}{\mv{I}}	
%  const. im Mathematikmodus							
\newcommand{\const}{\mathrm{const.}}			
%  imaginäre Einheit
\newcommand{\ii}{j}		
%  imaginäre Einheit * omega										
\newcommand{\jw}{\ii \, \omega}						
% <--------------- End of Mathematisches ----------------------

% --------------- Abkürzungen ------------------------------>
\newcommand{\zb}{\mbox{z.\,B.}\xspace}		%  Erzeugt z.B. mit korrektem Zwischenraum
\newcommand{\ua}{\mbox{u.\,a.}\xspace}		%  Erzeugt u.a. mit korrektem Zwischenraum
\newcommand{\ie}{\mbox{d.\,h.}\xspace}		%  Erzeugt d.h. mit korrektem Zwischenraum
\newcommand{\og}{\mbox{o.\,g.}\xspace}		%  Erzeugt o.g. mit korrektem Zwischenraum
\newcommand{\oa}{\mbox{o.\,a.}\xspace}		%  Erzeugt o.a. mit korrektem Zwischenraum
% <--------------- End of Abkürzungen ----------------------

\lhead{} 
\chead{} 
\rhead{} 

\lfoot{} \cfoot{} \rfoot{Seite \thepage\ von \pageref{LastPage}} 
\renewcommand{\headrulewidth}{0.4pt}
\renewcommand{\footrulewidth}{0.4pt}
\addto\captionsngerman{\renewcommand{\figurename}{Abb.}}
\addto\captionsngerman{\renewcommand{\tablename}{Tab.}}
%\captionsetup{justification = raggedright}

\begin{document}
\begin{titlepage}
	\newgeometry{left=2.5cm, right=2.5cm,top=2cm,bottom=0.1cm}

	%\begin{tikzpicture}[remember picture, overlay]
	%	\draw[line width = 1pt] ($(current page.north west) + (20mm,-15mm)$) rectangle ($(current page.south east) + (-20mm,15mm)$);
	%	\draw[line width = 1pt] ($(current page.north west)+ (20mm,-36mm)$) -- ($(current page.north west)+ (190mm,-36mm)$);
	%\end{tikzpicture}

	\begin{center}

		% Upper part of the page
		\begin{minipage}[t][2cm]{\textwidth}
			\begin{minipage}[b][1.5cm]{0.33\textwidth}
				\includegraphics[height=1.5cm]{assets/images/HZG_Logo_20_RGB}
			\end{minipage}
			\begin{minipage}[b][1.5cm][c]{0.55\textwidth}
				\vspace*{\fill}
				\centering
				\textbf{\Large Fakultät Elektrotechnik / Informatik}\\
				Fachbereich Elektrotechnik
				\vspace*{\fill}
			\end{minipage}
			\begin{minipage}[b][1.5cm]{0.1\textwidth}
				\includegraphics[height=1.4cm]{assets/images/fakultaetslogo_ohne_text.png}
			\end{minipage}
		\end{minipage}
		%\newline

		\sffamily
		\vspace{3cm}
		{\large\textbf{{Belegarbeit zur Lehrveranstaltung \\ Hins und Kunz}}}

		\vspace{3cm}
		zum Thema \\
		\vspace{0.5cm}
		{\Large\textbf{{Thema}}}

		\vspace{5cm}
		\textbf{Verfasser: } Verfasser \\
		\textbf{Seminargruppe: } Gruppe \\
		\textbf{Matrikelnummer: } Mat.-Nummer \\
		\textbf{Prüfer: } Mensch \\

		\vspace{3cm}
		\textbf{Abgabedatum: } Datum

	\end{center}

\end{titlepage}
\newgeometry{left=2.5cm, right=2.5cm,top=2.5cm,bottom=2.5cm}

%%%%%%%%%%%%%%%%%%%%%%%%%%%%%%%%%%%%%%%%%%%%%%%%%%%%%%%%%%%%%%%%%%%%%%%%%%%%%%%%
%%%%%%% START DOCUMENT  %%%%%%%%%%%%%%%%%%%%%%%%%%%%%%%%%%%%%%%%%%%%%%%%%%%%%%%%
%%%%%%%%%%%%%%%%%%%%%%%%%%%%%%%%%%%%%%%%%%%%%%%%%%%%%%%%%%%%%%%%%%%%%%%%%%%%%%%%
{
	\setstretch{1.0}
	\tableofcontents
}

\newpage
\section{Zielstellung}

Erläuterung des zu untersuchenden Sachverhalts.

\section{Einleitung}
\subsection{Unterabschnitt}

Gleichungsblock
\begin{align}
	C_n & = \frac{1}{T}\int_{-\frac{T}{2}}^{\frac{T}{2}}{f(t)e^{-jn\omega_0t}}dt                                                                                                                                                           \\
	    & = \dfrac{A}{T_0} \cdot \left[\dfrac{e^{-jn\omega_0t}}{-jn\omega_0t}\right]_{-\dfrac{\tau}{2}}^{\dfrac{\tau}{2}}                                                                                                                  \\
	    & = -\dfrac{A}{jn\omega_0 T}\cdot \left[e^{-jn\omega_0 \dfrac{\tau}{2}}-e^{jn\omega_0 \dfrac{\tau}{2}}\right]
	\intertext{Mit $e^{j\phi}=\cos (\varphi)+j \sin(\varphi)$ kann dieser Term vereinfacht werden:}
	C_n & = -\dfrac{A}{jn\omega_0 T}\cdot \left[\cos\left(-n\omega_0 \dfrac{\tau}{2}\right) + j\sin\left(-n\omega_0 \dfrac{\tau}{2}\right)-\cos\left(-n\omega_0 \dfrac{\tau}{2}\right)-j\sin\left(-n\omega_0 \dfrac{\tau}{2}\right)\right]
	\intertext{Mit $\cos(x) =\cos(-x)$ und $\sin(-x)=-\sin(x)$ kann weiter vereinfacht werden:}
	C_n & =-\dfrac{A}{jn\omega_0 T}\cdot \left[j\sin\left(-n\omega_0 \dfrac{\tau}{2}\right) + j\sin\left(-n\omega_0 \dfrac{\tau}{2}\right)\right]                                                                                          \\
	    & = -\dfrac{A}{jn\omega_0 T}\cdot \left[-2j\sin\left(n\omega_0 \dfrac{\tau}{2}\right) \right]
	\intertext{Die Kreisfrequenz $\omega_0$ kann ersetzt werden durch $\omega_0=2\pi f_0 = \frac{2 \pi}{T_0}$:}
	C_n & = \dfrac{A \cdot T}{2n\pi T}\cdot 2 \sin\left(\dfrac{2n\pi \tau}{2T}\right)                                                                                                                                                      \\
	    & = \dfrac{A}{n\pi}\cdot \sin\left(n\pi\frac{\tau}{T}\right)
	\intertext{Nach Erweiterung mit $\dfrac{\frac{\tau}{T}}{\frac{\tau}{T}}$}
	C_n & = \dfrac{A\cdot \tau}{T} \cdot \dfrac{\sin\left(\frac{n\pi \tau}{T}\right)}{\frac{n\pi \tau}{T}}
	\intertext{kann mit der si-Funktion vereinfacht werden:}
	C_n & = \dfrac{A\cdot \tau}{T} \cdot \text{si}\left({\frac{n\pi \tau}{T}}\right) \label{eq:koeff_si}
\end{align}

\section{Hauptteil}
\subsection{Zeug}
Der allgemeine Aufbau ist nachfolgend in Abbildung \ref{fig:1} dargestellt:

\begin{minipage}[t]{\textwidth-10pt}
	\centering
	\captionsetup{type=figure}
	\includegraphics[width=0.9\textwidth]{assets/images/Versuchsschema.png}
	\captionof{figure}{Schematischer Aufbau des Versuchs\cite{oA:2022:bk}}
	\label{fig:1}
\end{minipage}

\newpage
\section{Literaturverzeichnis}
\printbibliography
\newpage

\appendix
\section{Anhang}
\subsection{Messwerte}	
\vspace{0.5cm}
\begin{table}[h]
	\centering
	\captionof{table}{Messwerte, berechnete Fourier-Koeffizienten und Beträge der Übertragungsfunktion}
	\label{tab:mess}
	\begin{tabularx}{\textwidth}{cYYYYYc}
		\toprule
		\makecell{$\pmb{n}$\\$ $\\$ $} 
			& \makecell{$\pmb{U_{e}} /$\\$ \text{mV}$ \\ gemessen} 
			& \makecell{$\pmb{U_{a}} /$\\$ \text{mV}$ \\ gemessen}
			& \makecell{$\displaystyle \pmb{\frac{U_a}{U_e}}$\\$ $\\}
			& \makecell{$\pmb{U_e} /$\\$\text{mV}$ \\ berechnet}
			& \makecell{$\pmb{U_a} / $\\$\text{mV}$ \\ berechnet} 
			& \makecell{$\displaystyle \pmb{\frac{U_a}{U_e}} \Big \vert _{U_e=\hat{U}\sin 2 \pi \cdot n f_0}$\\$ $} 
		\\ \midrule
		\csvreader[
			head to column names,
			late after line=\\,
			late after last line=\\ \bottomrule]
			{assets/data/Messwerte.csv}{}{\1 & \2 & \3 & \4 & \5 & \6 & \7}
	\end{tabularx}
\end{table}
\end{document}