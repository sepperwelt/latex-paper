\section{Zielstellung}

Erläuterung des zu untersuchenden Sachverhalts.

\section{Einleitung}
\subsection{Unterabschnitt}

Gleichungsblock
\begin{align}
	C_n & = \frac{1}{T}\int_{-\frac{T}{2}}^{\frac{T}{2}}{f(t)e^{-jn\omega_0t}}dt                                                                                                                                                           \\
	    & = \dfrac{A}{T_0} \cdot \left[\dfrac{e^{-jn\omega_0t}}{-jn\omega_0t}\right]_{-\dfrac{\tau}{2}}^{\dfrac{\tau}{2}}                                                                                                                  \\
	    & = -\dfrac{A}{jn\omega_0 T}\cdot \left[e^{-jn\omega_0 \dfrac{\tau}{2}}-e^{jn\omega_0 \dfrac{\tau}{2}}\right]
	\intertext{Mit $e^{j\phi}=\cos (\varphi)+j \sin(\varphi)$ kann dieser Term vereinfacht werden:}
	C_n & = -\dfrac{A}{jn\omega_0 T}\cdot \left[\cos\left(-n\omega_0 \dfrac{\tau}{2}\right) + j\sin\left(-n\omega_0 \dfrac{\tau}{2}\right)-\cos\left(-n\omega_0 \dfrac{\tau}{2}\right)-j\sin\left(-n\omega_0 \dfrac{\tau}{2}\right)\right]
	\intertext{Mit $\cos(x) =\cos(-x)$ und $\sin(-x)=-\sin(x)$ kann weiter vereinfacht werden:}
	C_n & =-\dfrac{A}{jn\omega_0 T}\cdot \left[j\sin\left(-n\omega_0 \dfrac{\tau}{2}\right) + j\sin\left(-n\omega_0 \dfrac{\tau}{2}\right)\right]                                                                                          \\
	    & = -\dfrac{A}{jn\omega_0 T}\cdot \left[-2j\sin\left(n\omega_0 \dfrac{\tau}{2}\right) \right]
	\intertext{Die Kreisfrequenz $\omega_0$ kann ersetzt werden durch $\omega_0=2\pi f_0 = \frac{2 \pi}{T_0}$:}
	C_n & = \dfrac{A \cdot T}{2n\pi T}\cdot 2 \sin\left(\dfrac{2n\pi \tau}{2T}\right)                                                                                                                                                      \\
	    & = \dfrac{A}{n\pi}\cdot \sin\left(n\pi\frac{\tau}{T}\right)
	\intertext{Nach Erweiterung mit $\dfrac{\frac{\tau}{T}}{\frac{\tau}{T}}$}
	C_n & = \dfrac{A\cdot \tau}{T} \cdot \dfrac{\sin\left(\frac{n\pi \tau}{T}\right)}{\frac{n\pi \tau}{T}}
	\intertext{kann mit der si-Funktion vereinfacht werden:}
	C_n & = \dfrac{A\cdot \tau}{T} \cdot \text{si}\left({\frac{n\pi \tau}{T}}\right) \label{eq:koeff_si}
\end{align}

\section{Hauptteil}
\subsection{Zeug}
Der allgemeine Aufbau ist nachfolgend in Abbildung \ref{fig:1} dargestellt:

\begin{minipage}[t]{\textwidth-10pt}
	\centering
	\captionsetup{type=figure}
	\includegraphics[width=0.9\textwidth]{assets/images/Versuchsschema.png}
	\captionof{figure}{Schematischer Aufbau des Versuchs\cite{oA:2022:bk}}
	\label{fig:1}
\end{minipage}

\newpage
\section{Literaturverzeichnis}
\printbibliography
\newpage

\appendix
\section{Anhang}
\subsection{Messwerte}	
\vspace{0.5cm}
\begin{table}[h]
	\centering
	\captionof{table}{Messwerte, berechnete Fourier-Koeffizienten und Beträge der Übertragungsfunktion}
	\label{tab:mess}
	\begin{tabularx}{\textwidth}{cYYYYYc}
		\toprule
		\makecell{$\pmb{n}$\\$ $\\$ $} 
			& \makecell{$\pmb{U_{e}} /$\\$ \text{mV}$ \\ gemessen} 
			& \makecell{$\pmb{U_{a}} /$\\$ \text{mV}$ \\ gemessen}
			& \makecell{$\displaystyle \pmb{\frac{U_a}{U_e}}$\\$ $\\}
			& \makecell{$\pmb{U_e} /$\\$\text{mV}$ \\ berechnet}
			& \makecell{$\pmb{U_a} / $\\$\text{mV}$ \\ berechnet} 
			& \makecell{$\displaystyle \pmb{\frac{U_a}{U_e}} \Big \vert _{U_e=\hat{U}\sin 2 \pi \cdot n f_0}$\\$ $} 
		\\ \midrule
		\csvreader[
			head to column names,
			late after line=\\,
			late after last line=\\ \bottomrule]
			{assets/data/Messwerte.csv}{}{\1 & \2 & \3 & \4 & \5 & \6 & \7}
	\end{tabularx}
\end{table}